\vspace*{-15mm}
\begin{center}
  \hspace*{-12mm}
  \begin{tikzpicture}[thick,every text node part/.style={align=center}]

    % --------------------------------------------------------- %
    % ------------- principal node (question)
    % --------------------------------------------------------- %
    \node[text width=15cm] (question) at (10,10.25) {\textbf{\large Intégrer l'aménagement forestier et les interactions entre espèces au sein des modèles théoriques afin de mieux prédire la distribution et la productivité des espèces}};

    % --------------------------------------------------------- %
    % ------------ Objectives
    % --------------------------------------------------------- %
    % --- Obj0 node
    \node[text width=12cm,visible on=<2->] (obj0) at (10,8.75) {L'aménagement forestier peut-il augmenter le taux de migration de la forêt vers le nord ?};
    \draw [->,visible on=<2->] (question) edge [out=270, in=90] (obj0);
    % --- Obj1 node
    \node[text width=12cm,visible on=<3->] (obj1) at (10,7.25) {La stochasticité environnementale et démographique reste important quand on regarde sont effet à large echelles ?};
    \draw [->,visible on=<3->] (obj0) edge [out=270, in=90] (obj1);
    % --- Obj2 node
    \node[text width=9cm,visible on=<4->] (obj2) at (6.5,5.25) {Quand et comment les interactions biotiques sont plus importantes que le climat pour définir la limite de l'aire de répartition ?};
    \draw [->,visible on=<4->] (obj1) edge [out=195, in=90] (obj2);
    % --- Obj2 node
    \node[text width=8.25cm,visible on=<5->] (obj3) at (13.5,5.25) {Quand et comment l'aménagement forestier peut-il impacter la productivité et les limites d'aire de répartition ?};
    \draw [->,visible on=<5->] (obj1) edge [out=345, in=90] (obj3);
    % --- Obj3 node

  \end{tikzpicture}
\end{center}
